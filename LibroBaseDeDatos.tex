% ****************************************************************************************
% *********************       BASES DE DATOS                  ****************************
% ****************************************************************************************

% =======================================================
% =======         HEADER FOR DOCUMENT        ============
% =======================================================
    % *********   DOCUMENT ITSELF   **************
    \documentclass[12pt, fleqn]{report}                             %Type of docuemtn and size of font and left eq
    \usepackage[margin=1.2in]{geometry}                             %Margins and Geometry pacakge
    \usepackage{ifthen}                                             %Allow simple programming
    \usepackage{hyperref}                                           %Create MetaData for a PDF and LINKS!
    \hypersetup{pageanchor=false}                                   %Solve 'double page 1' warnings in build
    \setlength{\parindent}{0pt}                                     %Eliminate ugly indentation
    \author{Oscar Andrés Rosas}                                     %Who I am

    % *********   LANGUAJE AND UFT-8   *********
    \usepackage[spanish]{babel}                                     %Please use spanish
    \usepackage[utf8]{inputenc}                                     %Please use spanish - UFT
    \usepackage[T1]{fontenc}                                        %Please use spanish
    \usepackage{textcmds}                                           %Allow us to use quoutes
    \usepackage{changepage}                                         %Allow us to use identate paragraphs

    % *********   MATH AND HIS STYLE  *********
    \usepackage{ntheorem, amsmath, amssymb, amsfonts}               %All fucking math, I want all!
    \usepackage{mathrsfs, mathtools, empheq}                        %All fucking math, I want all!
    \usepackage{centernot}                                          %Allow me to negate a symbol
    \decimalpoint                                                   %Use decimal point

    % *********   GRAPHICS AND IMAGES *********
    \usepackage{graphicx}                                           %Allow to create graphics
    \usepackage{wrapfig}                                            %Allow to create images
    \graphicspath{ {Graphics/} }                                    %Where are the images :D

    % *********   LISTS AND TABLES ***********
    \usepackage{listings}                                           %We will be using code here
    \usepackage[inline]{enumitem}                                   %We will need to enumarate
    \usepackage{tasks}                                              %Horizontal lists
    \usepackage{longtable}                                          %Lets make tables awesome
    \usepackage{booktabs}                                           %Lets make tables awesome
    \usepackage{tabularx}                                           %Lets make tables awesome
    \usepackage{multirow}                                           %Lets make tables awesome
    \usepackage{multicol}                                           %Create multicolumns


    % *********   HEADERS AND FOOTERS ********
    \usepackage{fancyhdr}                                           %Lets make awesome headers/footers
    \pagestyle{fancy}                                               %Lets make awesome headers/footers
    \setlength{\headheight}{16pt}                                   %Top line
    \setlength{\parskip}{0.5em}                                     %Top line
    \renewcommand{\footrulewidth}{0.5pt}                            %Bottom line

    \lhead{                                                         %Left Header
        \hyperlink{chapter.\arabic{chapter}}                        %Make a link to the current chapter
        {\normalsize{\textsc{\nouppercase{\leftmark}}}}             %And fot it put the name
    }

    \rhead{                                                         %Right Header
        \hyperlink{section.\arabic{chapter}.\arabic{section}}       %Make a link to the current chapter
            {\footnotesize{\textsc{\nouppercase{\rightmark}}}}      %And fot it put the name
    }

    \rfoot{\textsc{\small{\hyperref[sec:Index]{Ve al Índice}}}}     %This will always be a footer  

    \fancyfoot[L]{                                                  %Algoritm for a changing footer
        \ifthenelse{\isodd{\value{page}}}                           %IF ODD PAGE:
            {\href{https://compilandoconocimiento.com/yo/}          %DO THIS:
                {\footnotesize                                      %Send the page
                    {\textsc{Oscar Andrés Rosas}}}}                 %Send the page
            {\href{https://compilandoconocimiento.com}              %ELSE DO THIS: 
                {\footnotesize                                      %Send the author
                    {\textsc{Compilando Conocimiento}}}}            %Send the author
    }
    
    
    
% ========================================
% ===========   COMMANDS    ==============
% ========================================

    % =====  GENERAL TEXT  ==========
    \newcommand \Quote {\qq}                                        %Use: \Quote to use quotes
    \newcommand \Over {\overline}                                   %Use: \Bar to use just for short
    \newcommand \ForceNewLine {$\Space$\\}                          %Use it in theorems for example
    
    \newenvironment{Indentation}[1][0.75em]                         %Use: \begin{Inde...}[Num]...\end{Inde...}
    {\begin{adjustwidth}{#1}{}}                                     %If you dont put nothing i will use 0.75 em
    {\end{adjustwidth}}                                             %This indentate a paragraph
    \newenvironment{SmallIndentation}[1][0.75em]                    %Use: The same that we upper one, just 
    {\begin{adjustwidth}{#1}{}\begin{footnotesize}}                 %footnotesize size of letter by default
    {\end{footnotesize}\end{adjustwidth}}                           %that's it


    % =====  GENERAL MATH  ==========
    \DeclareMathOperator \Space {\quad}                             %Use: \Space for a cool mega space
    \DeclareMathOperator \MiniSpace {\;}                            %Use: \Space for a cool mini space
    \newcommand \Such {\MiniSpace|\MiniSpace}                       %Use: \Such like in sets
    \newcommand \Also {\MiniSpace \text{y} \MiniSpace}              %Use: \Also so it's look cool
    \newcommand \Remember[1]{\Space\text{\scriptsize{#1}}}          %Use: \Remember so it's look cool

    \newtheorem{Theorem}{Teorema}[section]                          %Use: \begin{Theorem}[Name]\label{Nombre}...
    \newtheorem{Corollary}{Colorario}[Theorem]                      %Use: \begin{Corollary}[Name]\label{Nombre}...
    \newtheorem{Lemma}[Theorem]{Lemma}                              %Use: \begin{Lemma}[Name]\label{Nombre}...
    \newtheorem{Definition}{Definición}[section]                    %Use: \begin{Definition}[Name]\label{Nombre}...

    \newcommand{\Set}[1]{\left\{ \MiniSpace #1 \MiniSpace \right\}} %Use: \Set {Info}
    \newcommand{\Brackets}[1]{\left[ #1 \right]}                    %Use: \Brackets {Info} 
    \newcommand{\Wrap}[1]{\left( #1 \right)}                        %Use: \Wrap {Info} 
    \newcommand{\pfrac}[2]{\Wrap{\dfrac{#1}{#2}}}                   %Use: Put fractions in parentesis

    \newenvironment{MultiLineEquation}[1]                           %Use: To create MultiLine equations
        {\begin{equation}\begin{alignedat}{#1}}                     %Use: \begin{Multi..}{Num. de Columnas}
        {\end{alignedat}\end{equation}}                             %And.. that's it!
    \newenvironment{MultiLineEquation*}[1]                          %Use: To create MultiLine equations
        {\begin{equation*}\begin{alignedat}{#1}}                    %Use: \begin{Multi..}{Num. de Columnas}
        {\end{alignedat}\end{equation*}}                            %And.. that's it!


    % =====  LOGIC  ==================
    \DeclareMathOperator \doublearrow {\leftrightarrow}             %Use: \doublearrow for a double arrow
    \newcommand \lequal {\MiniSpace \Leftrightarrow \MiniSpace}     %Use: \lequal for a double arrow
    \newcommand \linfire {\MiniSpace \Rightarrow \MiniSpace}        %Use: \lequal for a double arrow
    \newcommand \longto {\longrightarrow}                           %Use: \longto for a long arrow

    % =====  NUMBER THEORY  ==========
    \DeclareMathOperator \Naturals  {\mathbb{N}}                     %Use: \Naturals por Notation
    \DeclareMathOperator \Primes    {\mathbb{P}}                     %Use: \Naturals por Notation
    \DeclareMathOperator \Integers  {\mathbb{Z}}                     %Use: \Integers por Notation
    \DeclareMathOperator \Racionals {\mathbb{Q}}                     %Use: \Racionals por Notation
    \DeclareMathOperator \Reals     {\mathbb{R}}                     %Use: \Reals por Notation
    \DeclareMathOperator \Complexs  {\mathbb{C}}                     %Use: \Complex por Notation

    % === LINEAL ALGEBRA & VECTORS ===
    \DeclareMathOperator \LinealTransformation {\mathcal{T}}        %Use: \LinealTransformation for a cool T
    \newcommand{\Mag}[1]{\left| #1 \right|}                         %Use: \Mag {Info} 

    \newcommand{\pVector}[1]{                                       %Use: \pVector {Matrix Notation} use parentesis
        \ensuremath{\begin{pmatrix}#1\end{pmatrix}}                 %Example: \pVector{a\\b\\c} or \pVector{a&b&c} 
    }
    \newcommand{\lVector}[1]{                                       %Use: \lVector {Matrix Notation} use a abs 
        \ensuremath{\begin{vmatrix}#1\end{vmatrix}}                 %Example: \lVector{a\\b\\c} or \lVector{a&b&c} 
    }
    \newcommand{\bVector}[1]{                                       %Use: \bVector {Matrix Notation} use a brackets 
        \ensuremath{\begin{bmatrix}#1\end{bmatrix}}                 %Example: \bVector{a\\b\\c} or \bVector{a&b&c} 
    }
    \newcommand{\Vector}[1]{                                        %Use: \Vector {Matrix Notation} no parentesis
        \ensuremath{\begin{matrix}#1\end{matrix}}                   %Example: \Vector{a\\b\\c} or \Vector{a&b&c}
    }

    % MATRIX
    \makeatletter                                                   %Example: \begin{matrix}[cc|c]
    \renewcommand*\env@matrix[1][*\c@MaxMatrixCols c] {             %WTF! IS THIS
        \hskip -\arraycolsep                                        %WTF! IS THIS
        \let\@ifnextchar\new@ifnextchar                             %WTF! IS THIS
        \array{#1}                                                  %WTF! IS THIS
    }                                                               %WTF! IS THIS
    \makeatother                                                    %WTF! IS THIS

    % TRIGONOMETRIC FUNCTIONS
    \newcommand{\Cos}[1]{\cos\Wrap{#1}}                             %Simple wrappers
    \newcommand{\Sin}[1]{\sin\Wrap{#1}}                             %Simple wrappers

    % === COMPLEX ANALYSIS ===
    \newcommand \Cis[1]  {\Cos{#1} + i \Sin{#1}}                    %Use: \Cis for cos(x) + i sin(x)
    \newcommand \pCis[1] {\Wrap{\Cis{#1}}}                          %Use: \pCis for the same ut parantesis
    \newcommand \bCis[1] {\Brackets{\Cis{#1}}}                      %Use: \bCis for the same to Brackets

    % === CALCULUS ===
    \newcommand \Partial[2] {\dfrac{\partial #1}{\partial #2}}      %Use: \Partial for simple use

    % =====  GENERAL COLOR  =========
    \definecolor{IndigoMD}{HTML}{3F51B5}                            %Use: Color :D
    \definecolor{DeepPurpleMD}{HTML}{673AB7}                        %Use: Color :D
    \definecolor{TealMD}{HTML}{009688}                              %Use: Color :D        
    \definecolor{BlueGrey800MD}{HTML}{37474F}                       %Use: Color :D
    \definecolor{BlueGrey100MD}{HTML}{CFD8DC}                       %Use: Color :D
    \definecolor{IndigoMD}{HTML}{3F51B5}                            %Use: Color :D
    \definecolor{Green100MD}{HTML}{DCEDC8}                          %Use: Color :D

    \newenvironment{ColorText}[1]{                                  %Use: \begin{ColorText}
        \leavevmode\color{#1}\ignorespaces}                         %That's is!


    % =====  CODE EDITOR =========
    \lstdefinestyle{CompilandoStyle} {                              %This is Code Style
        backgroundcolor=\color{BlueGrey800MD},                      %Background Color  
        basicstyle=\tiny\color{white},                              %Font color
        commentstyle=\color{BlueGrey100MD},                         %Comment color
        stringstyle=\color{TealMD},                                 %String color
        keywordstyle=\color{Green100MD},                            %keywords color
        numberstyle=\tiny\color{TealMD},                            %Size of a number
        frame=shadowbox,                                            %Adds a frame around the code
        breakatwhitespace=true,                                     %Style                       
        breaklines=true,                                            %Style                   
        keepspaces=true,                                            %Style                   
        numbers=left,                                               %Style                   
        numbersep=10pt,                                             %Style 
        xleftmargin=\parindent,                                     %Style 
        tabsize=4                                                   %Style 
    }
 
    \lstset{style=CompilandoStyle}                                  %Use this style




% =====================================================
% ============     	  COVER PAGE	   ================
% =====================================================
\begin{document}
\begin{titlepage}

	\center
	% ============ UNIVERSITY NAME AND DATA =========
	\textbf{\textsc{\Large Proyecto Compilando Conocimiento}}\\[1.0cm] 
	\textsc{\Large Programación}\\[1.0cm] 

	% ============ NAME OF THE DOCUMENT  ============
	\rule{\linewidth}{0.5mm} \\[1.0cm]
		{ \huge \bfseries Bases de Datos}\\[1.0cm] 
	\rule{\linewidth}{0.5mm} \\[2.0cm]
	
	% ====== SEMI TITLE ==========
	{\LARGE Una Pequeña (Gran) Introducción}\\[7cm] 
	
	% ============  MY INFORMATION  =================
	\begin{center} \large
    \textbf{\textsc{Autores:}}\\
        Rosas Hernandez Oscar Andrés \\
        Lopez Manriquez Angel
    \end{center}

	\vfill

\end{titlepage}

% =====================================================
% ========                INDICE              =========
% =====================================================
\tableofcontents{}
\label{sec:Index}

\clearpage




% //////////////////////////////////////////////////////////////////////////////////////////////////////////
% ///////////////////////////////////         PARTE ABSTRACTA              /////////////////////////////////
% //////////////////////////////////////////////////////////////////////////////////////////////////////////
\part{Parte Abstracta}
\clearpage


    % ===============================================================================
    % ===================           DEFINICIONES               ======================
    % ===============================================================================
    \chapter{Definiciónes}

        % ==============================================
        % ========    REPOSITORIO DE DATOS     =========
        % ==============================================
        \clearpage
        \section{Repositorio de Datos}
            
            Son un conjunto de datos, donde definio a un dato como cualquier información
            que sea váliosa.


        % ==============================================
        % ========        BASE DE DATOS        =========
        % ==============================================
        \section{Base de Datos}

            Son un conjunto de datos interrelacionado definido por un modelo de datos, esto no lo
            tiene necesariamente un repositorio de datos. Así como progrmas que nos permitan acceder
            y manipular esa información.

            Podemos definir de manera alterna como: Una colección de registros el cual es almacenada
            en una computadora de una forma sistemática (estructurada), de tal forma que un programa
            de computadora pueda consultarlo para responder consultas.



        % ==============================================
        % =======    RAZONES O PROPOSITOS       ========
        % ==============================================
        \section{Proposito para una Base de Datos}

            Todo muy bien, pero ¿porque debería importarme un comino?¿Porqué preferir una base de datos
            sobre simplemente guardar los datos de manera \Quote{común}?

            Antes de la aparición de los SGBD, las organizaciones normalmente almacenaban la
            información en Sistemas de Procesamiento de Archivos Típicos (Sistemas de Archivos).

            Un sistema de archivos es un conjunto de programas que prestan servicio a los usuarios
            finales, donde cada programa define y maneja sus propios datos, los cuales presentan
            los siguientes inconvenientes:

            \begin{itemize}
                \item
                    \textbf{Redundancia de datos e inconsistencia}:
                        Ya que tu no vas a programar un sistema entero habra muchas maneras
                        en que los demás programadores crearán aplicaciones y sobretodo en como
                        van a guardar los datos.

                        Peor aún, ¿Qué pasa cuando tengamos un motón de archivos con casi la misma
                        información?
                        Es decir cuando tengamos un montón de archivos con tu mismo número de telefono,
                        con tu misma información de contacto. 

                \clearpage

                \item
                    \textbf{Problemas para acceder a la información que queriamos en primer lugar}:
                        Supongamos que queremos acceder a los datos, digamos que tenemos un montón
                        de registros de sobre alumnos.

                        ¿Como hariamos para tener todos los alumos que hayan reprobado?
                        No hay forma fácil de hacerlo, incluso la forma mas \Quote{correcta} sería
                        desarrollar un pequeño programa que se encargue de hacer lo que queremos.

                        Y esto podría servir muy bien.... Hasta que necesitemos algo más.
                        Entonces tenemos que estar haciendo programas y programas ¡Que cansado!

                \item
                    \textbf{Aislamiento de la Información}:
                        Ya que puede estar repartida por todos lados, no se exactamente por donde tengo
                        que empezar a buscar. Y esto se vuelve un verdadero desastre cuando intentamos
                        modificar la información guardada.

                \item
                    \textbf{Problemas con Integridad}:
                        Los valores de los datos almacenados en la base de datos, deben satisfacer
                        ciertos tipos de restricciones de consistencia.

                        Por ejemplo, el saldo de cierto tipo de cuentas bancarias no pueden ser nunca
                        inferior a una cantidad predeterminada, digamos 4000 dolares.

                        Los desarrolladores deben cumplir estas restricciones en el sistema añadiendo
                        el código correspondiente en los diversos programas de aplicación. Sin embargo,
                        cuando se añaden nuevas restricciones, es difícil cambiar los programas para
                        hacer que se cumplan. 

                \item
                    \textbf{Problemas con Atomicidad}:
                        Si ocurriera un problema en el sistema al momento de estar modificando la información
                        me gustaria que al volver a arrancar todo, este debería regresar a un punto de
                        respaldo.

                        Como si el sistema fallara justo al hacer una transacción bancaria, me interesa que 
                        se haya realizado o no, pero que no le haya quitando dinero a una cuenta pero no se la
                        haya dado al otro cliente. 


                \item
                    \textbf{Errores del Acceso Concurrente}:
                        Para aumentar el rendimiento global del sistema y obtener una respuesta más
                        rápida, muchos sistemas permiten que varios usuarios actualicen los datos
                        simultaneamente. En realidad hoy en día, los principales sitios de comercio
                        electrónico en internet pueden tener millones de accesos diarios de compradores
                        a sus datos. En tales entornos es posible la interacción de actualizaciones
                        concurrentes y puede dar lugar a datos inconsistentes.


                \item
                    \textbf{Problemas de seguridad}:
                        No todos los usuarios de un sistema de base de datos deben tener acceso a todos
                        los datos. Ya que los programas de aplicación se añaden al sistema de
                        procesamiento de datos de un forma adhoc, es difícil hacer cumplir tales
                        restricciones de seguridad.

            \end{itemize}

            \clearpage

            Podemos resumir esto con que a diferencia de un sistema de archivos, una base de datos busca:
            \begin{itemize}
                \item Evitar o miniza la redundancia
                \item Evitar inconsistencias en los datos
                \item Eliminar inconsistencias en los datos
                \item Comunicacion con distintos repositorios de datos
                \item Control de concurrencia
            \end{itemize}




        % ==============================================
        % ===    ELEMENTOS DE UNA BASE DE DATOS   ======
        % ==============================================
        \clearpage
        \section{Elementos de un Sistema de Base de Datos}

            \includegraphics[width=0.85\textwidth]{DiagramaPartes.png}

            \begin{itemize}
                \item
                    \textbf{Aplicaciones:} Es la interfaz entre la base de datos y el usuario, estas pueden
                    ser desarrolladas por un lenguaje de alto nivel 

                \item
                    \textbf{Conector:} Son los componentes que permiten el enlace entre el SGBD y las
                    interfaces desarrolladas en un lenguaje de programación, estas contienen las clases
                    y/o funciones necesarias para llevar a cabo la comunicación entre las aplicaciones
                    con el Sistema Gestor de Base de Datos.

                \item
                    \textbf{Sistema Gestor de Base de Datos:} Son el software especialízado que
                    nos permite manipular inteligentemente nuestro datos:

                    Es la aplicación que permite a los usuarios definir, crear y mantener la base
                    de datos y proporciona acceso controlado a la misma.

                    \begin{itemize}
                        \item Creación de Repositorios
                        \item Creación de Cuentas de Usuarios
                        \item Se encarga de crear archivos lógicos, físicos y objetos de la BD.
                        \item Se encarga de administrar las transacciones, bloqueos , etc.
                    \end{itemize}

            \end{itemize}




        % ==============================================
        % ========   ABSTRACCIÓN DE LOS DATOS   ========
        % ==============================================
        \clearpage
        \section{Abstracción de los Datos}

            \begin{itemize}
                \item
                    \textbf{Nivel Físico}: 

                    Es el nivel base de abstracción que describe como es que la información
                    es guardade de manera actual. Esto nos permite describir como de complejo
                    es que son las estructuras en la realidad.

                \item
                    \textbf{Nivel Lógico}: 

                    Es el nivel que nos muestra como es que se almacena la información dentro de
                    la base de datos y como es que se da la relaciones entre la información.

                    Aunque la implementación de las estructuras simples en el nivel lógico puede
                    implicar complejas estructuras de nivel físico, el usuario de este nivel no
                    necesita ser consciente de esta complejidad.
                    Esto se conoce como independencia de datos físicos. 
                    Los administradores de bases de datos, que deben decidir qué información debe
                    conservar en la base de datos, utilizan el nivel lógico de abstracción.

                \item
                    \textbf{Nivel Visual}: 

                    El nivel más alto de abstracción describe sólo una parte de la base de datos completa.

                    Aunque el nivel lógico utiliza estructuras más simples, la complejidad se mantiene
                    debido a la variedad de información almacenada en una base de datos grande.

                    Muchos usuarios del sistema de base de datos no necesitan toda esta información, 
                    sólo necesitan acceder a una parte de la base de datos.
                    El nivel de vista de la abstracción existe para simplificar su interacción
                    con el sistema.

            \end{itemize}



        % ===============================================
        % ====     USUARIOS DE UNA BASE DE DATOS     ====
        % ===============================================
        \clearpage
        \section{Usuarios de una Base de Datos}


            Hay cuatro grupos de personas que intervienen en el entorno de un sistema de base de datos:
            el administrador de la base de datos, los diseñadores de la base de datos, los programadores
            de aplicaciones y los usuarios.

            \begin{itemize}

                \item
                    \textbf{Diseñadores de la Base de Datos}

                    \begin{itemize}
                        \item Encargado de grabar los propios Módelos de Datos
                        \item Esquema de la Base de Datos
                        \item Diseño lógico de la Base de Datos
                    \end{itemize}


                \item
                    \textbf{Administrador de la Base de Datos}

                    Encargado de: 

                        \begin{itemize}
                            \item Monitorear el Performance
                            \item Diseño físico de la base de datos y de su implementación.
                            \item Herramientas Administrativas
                                \begin{itemize}
                                    \item Creacion de cuentas de usuarios
                                    \item Objetos accedidos.
                                    \item Matriz de autorizacion.
                                \end{itemize}
                            \item Definir tiempos de respaldo
                            \item Reorganización físico
                            \item Llevar a cabo las tecnicas de recuperación
                        \end{itemize}

                \item
                    \textbf{Programadores de Aplicaciones}

                    Son los que se encargan de implementar los programas de aplicación que
                    servirán a los usuarios finales. Estos programas son los que permiten
                    consultar datos, insertarlos, actualizarlos y eliminarlos.

                    \begin{itemize}
                        \item Interfaces de los usuarios finales
                        \begin{itemize}
                            \item Facilitar el acceso a ciertos "objetos" de la BD
                        \end{itemize}

                        \item Interfaces para la gestión de la aplicaciones 
                        \begin{itemize}
                            \item Operaciones de escritura sobre altas ó bajas 
                            \item IDE desarrollo (Java, .NET)
                            \item Lenguaje Scripts
                            \item Conectividad servidores de datos (API s) 
                        \end{itemize}
        
                    \end{itemize}


            \end{itemize}


            % ==================================
            % ====     USUARIOS FINALES     ====
            % ==================================
            \clearpage
            \subsection{Usuarios Finales}

                Los usuarios finales son los clientes de la base de datos, son las personas que
                requieren acceso a la base de datos para realizar consultas, actualizaciones e
                informes. Los usuarios se pueden clasificar en varias categorías:
                \begin{itemize}
                        \item
                            \textbf{Casuales}
                            Estos acceden ocasionalmente a la base de datos, pero pueden necesitar una
                            información diferente en cada momento.

                            \begin{itemize}
                                \item Data Minning
                                \item Big Data
                            \end{itemize}

                        \item 
                            \textbf{Principiantes - Paramétricos}
                            Constituyen una parte considerable de los usuarios finales de los
                            sistemas de bases de datos. Su labor principal gira entorno a la
                            consulta y actualización constantes de la BD.

                        \item 
                            \textbf{Sofisticados}
                            \begin{itemize}
                                \item Experiencia en aplicaciones 
                                \item Programadores de aplicación
                                \item DBA 
                                \item Investigadores
                             \end{itemize}
                            
                        \item 
                            \textbf{Independientes (Stand Alone)} 
                                \begin{itemize}
                                    \item Sistemas Escolares
                                    \item Sistemas de prueba
                                    \item Sin conectividad a otros nodos
                                \end{itemize}

                \end{itemize}






        % ===============================================
        % ===========     DML VS DDL      ===============
        % ===============================================
        \clearpage
        \section{DML vs DDL}

            \begin{itemize}

                \item
                    \textbf{DML}

                        Data Manipulation Languaje, es un lenguaje que nos permite modificar los
                        datos guardados.

                        Los tipos de acceso que tenemos disponibles son:

                        \begin{itemize}
                            \item Recuperar información
                            \item Insertar nueva información
                            \item Eliminar la información
                            \item Modificación de la información
                        \end{itemize}

                    Una consulta o query es una sentencia que solicita la recuperación de información.
                    La parte de un DML que implica recuperación de información se denomina lenguaje
                    de consulta.

                    Aunque técnicamente incorrecto, solemos utilizar los términos lenguaje de consulta
                    y lenguaje de manipulación de datos como sinónimos.


                \item
                    \textbf{DDL}

                    Data Definition Languaje, es Lenguaje de definición de datos.

                    Especificamos un esquema de base de datos mediante un conjunto de definiciones
                    expresadas por un lenguaje especial denominado DDL.

                    El DDL también se utiliza para especificar propiedades adicionales de los datos.

            \end{itemize}
            \clearpage


            Podemos simplificar en el sentido de SQL que:

            \textbf{DDL: Data Definition Language}, mediante el cual puede definir nuevos objetos
            de base de datos, como Table, Views, Stored Procedures, etc.
            Algunos comandos comunes son:

            \begin{itemize}
                \item CREATE
                \item ALTER
                \item DROP
                \item etc...
            \end{itemize}


            \textbf{DML: Data Manipulation Language}, mediante el cual puede realizar cambios en
            los objetos creados anteriormente por DDLs. Algunos comandos comunes son:

            \begin{itemize}
                \item INSERT
                \item UPDATE
                \item DELETE
                \item SELECT
                \item etc...
            \end{itemize}




        % ===============================================
        % ====     ARQUITECTURA DE SISTEMA DE DB    =====
        % ===============================================
        \clearpage
        \section{Arquitectura del Sistema Gestor}

            Podemos acceder al Sistema Gestor de nuestra Base de Datos de muchas maneras, 
            desde formularios web, aplicaciones de escritorio e interpretes de SQL, todos se
            comunican con la Base de Datos mediante SQL.

            \begin{itemize}

                \item
                    \textbf{Motor de Evaluación de Consultas}

                        \begin{itemize}
                            \item Analizador:
                                Este se encarga de analizar las sentencias SQL a nivel
                                sintactico y lexico, así como una validación de que existan
                                dichas relaciones y atributos.

                                Es lo que mucha gente conoce como un compilador de DDL ó DML.

                            \item Evaluador de Operaciones:
                                Este se encarga de crear el árbol canónico, es decir, la forma
                                formal que tendría el árbol necesario para acceder a la información.

                            \item Optimizador:
                                Se encarga de tomar el árbol canónico y optimizarlo para que se tenga
                                que hacer la menor cantidad de operación a nivel lógico, este se conoce
                                como árbol de consulta. 

                            \item Ejecutor de Planes:
                                Se encarga de planear la mejor estrategia para ejecutar la consulta
                                de tal manera que se optimize de manera física.
                        \end{itemize}

                \clearpage
                \item
                    \textbf{Gestores}

                        \begin{itemize}
                            \item Gestor de Transacciones:
                                Este es el que se encarga de organizar varias sentencias SQL para ejecutarlas
                                como una transacción.
                            
                            \item Gestor de Bloqueos:
                                Este es el que se encarga de ver si es que cierta relación 
                                esta bloqueada porque esta ocurriendo una transacción en ese momento.

                                Los bloqueos son una parte muy importatne de este gesto.

                                Podemos definirlos en principalmente 2:
                                \begin{itemize}
                                    \item Binario:
                                        Es decir permite que una relación este o bine bloqueada
                                        o no para lectura o escritura.
                                    \item Múltiplos Niveles:
                                        Es decir nos permite que existan bloqueos de lectura,
                                        escritura diferentes, ayudando a evitar las esperas si no
                                        son necesarias.
                                \end{itemize}

                            \item Gestor de Recuperación:
                                Este es el que se encarga de definir todas las técnicas de recuperación
                                sobre la base, estas mismas las podemos definir en dos:
                                \begin{itemize}
                                    \item Inmediatas: Son las que podemos representar como rollback o rollfoward
                                    \item Técnicas en Frío: Usa una copia de seguridad
                                \end{itemize}


                            \item Gestor de Archivos y Métodos de Acceso
                                Todos los elementos, metadatos, logs, funciones, etc... tienen que estar
                                almacenada de manera física.

                                Este se encarga de dependiendo de las querys como es que debe acceder de 
                                manera eficiente a los datos.

                            \item Gestor de Memoria Intermedia
                                Es como una RAM, nos permite guardar consultas bastante frecuentes
                                para aumentar la velocidad sobre la RAM
                        \end{itemize}

            \end{itemize}


            \begin{figure}[h!]
                \centering
                \includegraphics[width=0.65\textwidth]{ArquitecturaSGBD.pdf}
            \end{figure}







        % ===============================================
        % ====     ARQUITECTURA DE BASES DE DATOS   =====
        % ===============================================
        \clearpage
        \section{Arquitecturas de las Bases de Datos}


            \begin{itemize}

                \item
                    \textbf{Cliente - Servidor}

                    Mientras que el servidor es aquel que contiene toda la información 
                    importante, el cliente solo se encarga de solicitar servicios.

                    Podemos seperar estos como:
                    \begin{itemize}
                        \item Cliente sin Disco:
                            Aquel que puede acceder a la información pero no puede modificarlo.

                        \item Cliente con Disco:
                            Aquel que puede acceder a la información y puede modificarlo.
                     \end{itemize} 


                \item
                    \textbf{Multibase}

                    Esta se caracterisa porque cada base de datos interna que tienes no se relaciona
                    entre si, es decir, no hay comunación entre ellas.

                        
            \end{itemize}




    % ===============================================================================
    % ====================     MODELO ENTIDAD RELACION      =========================
    % ===============================================================================
    \clearpage
    \chapter{Sistema Entidad-Relación}


        % ==============================================
        % ================  QUE ES   ===================
        % ==============================================
        \clearpage
        \section{¿Qué es?}

            Fue creado por Peter Pin‐Shan Chen en 1976, llamado \Quote{The Entity Relationship
            Model Toward Unified View of Data}

            Es un modelo conceptual, es decir, creado en lenguaje natural y se se ayuda 
            de DER (gráficas) para hablar de las relaciones.

            Permite construir el modelo conceptual de datos.
            \begin{itemize}
                
                \item Es una representación de la estructura y contenido de una base de datos.

                \item Es independiente  del software (como el SMBD).
                
                \item Permite establecer las restricciones de la BD.

                \item Esta asociado al modelo de datos que es usado para implementar la BD.

            \end{itemize}

            Siendo una publicado por la ACM (Association for Computing Machinery), 
            considera los siguientes puntos:

            \begin{itemize}
                \item Incorpora información semántica del mundo real.

                \item Introduce una técnica gráfica como herramienta para el diseño
                    de bases de datos (Diagrama Entidad Relación).

                \item Es una representación lógica de los datos de una organización o
                    un área de negocio.

                \item Normalmente es expresado como un DER, siendo este la representación
                    gráfica del Modelo ER.
            \end{itemize}

            \begin{figure}[h]
                \centering
                \includegraphics[width=0.50\textwidth]{ER-Diagram}
                \caption{Ejemplo de un Diagrama Entidad-Relación}
            \end{figure}



        % ==============================================
        % =========          ENTIDADES         =========
        % ==============================================
        \clearpage
        \section{Entidades}
                
            Son objetos que existen en el mundo real y que son distingubles
            (gracias a sus características) de otros objetos.

            El tipo de entidad es el esquema que tiene la entidad en la base de datos,
            es decir son todos los atributos o características que definien a la entidad.

            Piensa en las entidades como sustantivos.
            Ejemplos: cliente, estudiante, automóvil o producto. 


            \begin{figure}[h]
                \centering
                \includegraphics[width=0.30\textwidth]{Entidad}
                \includegraphics[width=0.30\textwidth]{EntidadDebil}
                \includegraphics[width=0.30\textwidth]{EntidadAsociativa}
            \end{figure}


            \begin{itemize}
                \item \textbf{Fuerte:}\\
                    Es aquella entidad cuya existencia no depende de otras entidades.
                \item 
                    \textbf{Debíl:}\\
                    Es aquella entidad cuya existencia depende de otras entidades.
                    \begin{itemize}
                        \item Esta depende de de una o varias entidades fuertes
                        \item Una entidad debíl jamas se debe asociar con una entidad debíl
                        \item Una entidad debíl no tiene identificador propio, pero si parciales
                        \item Usa una relación identificada para asociar una entidad fuerte con una debíl
                    \end{itemize}

                    \begin{figure}[h]
                        \centering
                        \includegraphics[width=0.80\textwidth]{EjemploEntidadDebil}
                    \end{figure}

                \clearpage

                \item \textbf{Asociativa:}\\
                    Se usa cuando tienes una relación y resulta que quieres usar esta relación
                    como conexión (por ejemplo la relación entre alumno y clase) así que haces
                    esa relación una entidad asociativa (por ejemplo para asociar esta nueva
                    entidad con profesor).

                    Relación Identificada: Es la relación entre una entidad débil y el
                    identificador propio.

                    \begin{figure}[h]
                        \centering
                        \includegraphics[width=0.80\textwidth]{EjemploEntidadAsociativa}
                        \caption{Ejemplo}
                    \end{figure}

                    \vspace{5em}

                    \begin{figure}[h]
                        \centering
                        \includegraphics[width=0.80\textwidth]{EjemploEntidadAsociativa2}
                        \caption{Otro Ejemplo}
                    \end{figure}
            \end{itemize}



        % ==============================================
        % =========      RELACIONES            =========
        % ==============================================
        \clearpage
        \section{Relaciones}
                
            Esto es como las entidades actúan unas sobre otras o están asociadas entre sí.
            Piensa en las relaciones como verbos.

            Por ejemplo, el estudiante nombrado puede registrarse para un curso.
            Las dos entidades serían el estudiante y el curso, y la relación representada es
            el acto de inscribirse, conectando las dos entidades de esa manera. 

            \begin{figure}[h]
                \centering
                \includegraphics[width=0.30\textwidth]{Relacion}
                \includegraphics[width=0.30\textwidth]{RelacionIdentificada}
            \end{figure}


        % ==============================================
        % =========      CARDINALIDAD          =========
        % ==============================================
        \subsection{Cardinalidad}

            Nos dice la cantidad de atributos que se relacionan.


            \begin{figure}[h]
                \centering
                \includegraphics[width=0.75\textwidth]{Cardinalidad}
            \end{figure}

        % ==============================================
        % =======   ATRIBUTOS EN RELACION      =========
        % ==============================================
        \clearpage
        \subsection{Atributos en las Relaciones}

            Atributos en relaciones: Los atributos pueden ser asociados con relaciones
            como una entidad.

            En este caso, se requiere almacenar la Fecha (mes y año) cuando un empleado
            completa un curso.


            \begin{figure}[h]
                \centering
                \includegraphics[width=0.80\textwidth]{RelacionConAtributos}
            \end{figure}

        % ==============================================
        % =======     GRADO DE UNA RELACIÓN    =========
        % ==============================================
        \subsection{Grado de una Relación}

            El número de entidades que participan en una relación.
            Las Relaciones de acuerdo al grado de relación

            \textbf{Relaciones Unarias: }

                Cuando una única relación interviene en la asociación

                \begin{figure}[h]
                    \includegraphics[width=0.20\textwidth]{RelacionUnaria}
                    \includegraphics[width=0.20\textwidth]{RelacionUnaria2}
                \end{figure}


        % ==============================================
        % =========      ATRIBUTOS             =========
        % ==============================================
        \clearpage
        \section{Atributos}

            Son las propiedades o características de una entidad
            \begin{itemize}

                \item \textbf{Simples:}
                    Son aquelllas propiedades atomicas que ya no se pueden subdividir
                    
                    \includegraphics[width=0.20\textwidth]{Atributo}

                \item \textbf{Identificador:}
                    Es aquel atributo que nos permite identificar de forma unica cada una de las
                    instancias de la entidad.

                    \includegraphics[width=0.20\textwidth]{AtributoID}

                \item \textbf{Identificador Parcial:}

                    \includegraphics[width=0.20\textwidth]{AtributoIDParcial}

                \item \textbf{Derivado:}
                    Es aquel atributo cuyo valor puede ser calculado a partir de
                    otros valores de atributos (posiblemente de datos que no se
                    encuentren en la BD's, tal como, la fecha del sistema, etc)

                    \includegraphics[width=0.20\textwidth]{AtributoDerivado}

                \item \textbf{Multivalor:}
                    Aquellos que tienen mas de un valor

                    \includegraphics[width=0.20\textwidth]{AtributoMultiValor}


                \item \textbf{Compuestos:}
                    Aquellos que estan formados por varios subatributos

                    \includegraphics[width=0.30\textwidth]{AtributoCompuesto}

            \end{itemize}





    % ========================================================
    % ========    MODELO EXTENDIDO ENTIDAD RELACION    =======
    % ========================================================
    \clearpage
    \chapter{Sistema Entidad-Relación Extendido}



        % =========================================
        % ========    INTRODUCCIÓN    =============
        % =========================================
        \clearpage
        \section{Introducción}

            \begin{wrapfigure}{r}{0.18\textwidth}
                \centering
                \includegraphics[width=0.16\textwidth]{EERD0}
            \end{wrapfigure}

            El Modelo de Relación de Entidad Extendida es un modelo más complejo y de más
            alto nivel que extiende un diagrama E-R para incluir más tipos de abstracción
            y para expresar más claramente las restricciones.

            Todos los conceptos de un diagrama E-R están incluidos en el modelo EE-R.

            Vamos a ver las diferencias más importantes con respecto a nuestro clásico
            amigo el diagrama entidad relación.

        % =========================================
        % ========    CARACTERISTICAS    ==========
        % =========================================
        \begin{itemize}
            \item
                \textbf{Herencia: Super y Subclases}

                Como el nombre sugiere, podemos tener entidades que consideramos padres
                y que heredan a otras entidades hijas.

                Por ejemplo podemos podemos tener una entidad empleado puede tener
                subentidades que se dividen por el trabajo que hacen, por ejemplo
                diseñadores, gerentes y ayudantes.

                Todas nuestras subentidades van a heredar las propiedades o atributos
                de nuestra superentidad.

                Esto nos permite que nuestra entidad padre sea una idea general y que
                nuestras subclases sean especializaciones de la misma, por esto decimos
                que esta clase de diagramas permite la especialización.


                Es también importante remarcar que alguna de nuestras entidades
                puede heredar de una o mas entidades


            \item
                \textbf{Generalización y Especialización}

                Como habiamos hablando antes, esto nos permite que si tenemos
                una gran cantidad de entidades con atributos comunes podemos 
                crear una entidad general y hacer que las demás hereden de ella,
                con esto también propiciamos que nuestras entidades que heredan 
                se puedan especializaciar.


            \clearpage

            \item
                \textbf{Restricciones}
                    Tambien es común que se les conozca como superclass constraint

                    \begin{itemize}
                        \item Entidades Disconjuntas
                        \item Entidades Superpuestas
                    \end{itemize}

            \item
                \textbf{Discriminadores de Superentidad}

                Un discriminador de entidades es un atributo que indica el subtipo
                de una entidad.

                Es decir podemos indicar con este diagrama si las divisiones que hacemos
                son parciales (es decir, es posible que una instancia de la sueprclase
                pueda existir sin tener que ser parte de una subentidad) o total
                en la que decimos que cualquier instancia tiene que pertencer a alguna
                subentidad

                \begin{figure}[h]
                    \centering
                    \includegraphics[width=0.30\textwidth]{EERD0}
                    \includegraphics[width=0.30\textwidth]{EERD1}
                \end{figure}

        \end{itemize}







    % ===============================================================================
    % ===================        SISTEMA RELACIONAL         =========================
    % ===============================================================================
    \chapter{Sistema Relacional}


        % ==============================================
        % ===============   INTRODUCCION     ===========
        % ==============================================
        \clearpage
        \section{Introducción}

            El modelo Relacional fue introducido en 1970 por E.F. Codd.

            La mayoría de los sistemas de bases de datos utilizados hoy en día son relacionales.


        % ==============================================
        % =========   DEFINICION BASICA      ===========
        % ==============================================
        \clearpage
        \section{Definiciones Básicas}

            En el modelo relacional, los datos se dividen en diferentes \textbf{tablas}.
            Una tabla funciona como una matriz o una hoja de cálculo.

            \begin{figure}[h]
                \centering
                \includegraphics[width=0.95\textwidth]{EjemploTabla}
            \end{figure}

            \begin{itemize}
                \item \textbf{Relación}: Una tabla con columnas y filas
                \item \textbf{Atributo}: Nombre de una columna de una relación
                \item \textbf{Dominio}: Es el conjunto de valores legales de el atributos
                \item \textbf{Tupla}: Es una fila de una relación
                \item \textbf{Grado de una Relación}: Es el número de atributos que contiene la relación
                \item \textbf{Cardinalidad de una Relación}: Es el número de tuplas que contiene
            \end{itemize}




        % ==============================================
        % =========      RELACIONES            =========
        % ==============================================
        \clearpage
        \section{Relaciones}

            Para evitar la redundancia de datos la clave esta en crear diversas tablas
            e ir relacionandolas.

            Las relaciones tienen las siguientes características:

            \begin{itemize}
                \item Cada tupla dentro de la tabla tiene que ser única
                \item El orden de las tuplas no importa
                \item Cada relación tiene un nombre único en la Base de Datos
                \item Los atributos de cada tabla tienen que tener un nombre único en la tabla
                \item No importa el orden de los atributos
                \item Los valores de los atributos son atómicos en cada tupla, es decir,
                    cada atributo toma un solo valor. Se dice que las relaciones están normalizadas.
            \end{itemize}

            % ==============================================
            % ==========     VALORES NULOS       ===========
            % ==============================================
            \subsubsection{Valores Nulos}

                Cuando en una tupla un atributo es desconocido, se dice que es nulo.
                    
                Un nulo no representa el valor cero ni la cadena vacía, éstos son valores que
                tienen significado.

                El nulo implica ausencia de información, bien porque al insertar la tupla se 
                desconocía el valor del atributo, o bien porque para dicha tupla el atributo no
                tiene sentido.

            % ==============================================
            % ==========   LLAVES RELACIONES     ===========
            % ==============================================
            \clearpage
            \subsection{Llaves Relaciones}

                Para lograr esto tenemos que crear relaciones, y es imposible hablar de
                relaciones sin hablar de llaves:

                \begin{itemize}
                    \item
                        \textbf{Primary Key:}
                        La idea de una llave primaria es crear un atributo (o un conjunto de 
                        atributos) que tiene, tiene pero tiene que ser único y es recomendable
                        que casi no cambie.

                        Una llave compuesta es una llave primaria que consiste de más de un atributo

                        \includegraphics[width=0.55\textwidth]{LlavesPrimarias}

                    \item
                        \textbf{Foreign Key:}
                        La idea de una llave secundaria es crear un atributo (o mas comunmente
                        varios atributos) que en alguna otra tabla sea la llave primaria. Eso es todo.

                        \includegraphics[width=0.55\textwidth]{LlaveForanea}

                \end{itemize}


            % ==============================================
            % ==========   REGLAS DE INTEGRIDAD  ===========
            % ==============================================
            \clearpage
            \subsection{Reglas de Integridad}

                Una vez definida la estructura de datos del modelo relacional, pasamos a estudiar
                las reglas de integridad que los datos almacenados en dicha estructura deben
                cumplir para garantizar que son correctos.

                El definir cada atributo sobre un dominio se impone una restricción sobre el
                conjunto de valores permitidos para cada atributo. A este tipo de restricciones
                se les denomina restricciones de dominios.

                Hay además dos reglas de integridad muy importantes que son restricciones que se
                deben cumplir en todas las bases de datos relacionales y en todos sus estados o
                instancias (las reglas se deben cumplir todo el tiempo). 


                \begin{itemize}
                    \item
                        \textbf{Ninguno de los atributos que componen la Llave Primaria
                        puede ser nunca nulo}

                        Por definición, una clave primaria es un identificador irreducible que se 
                        utiliza para identificar de modo único las tuplas, si se permite que parte
                        de la clave primaria sea nula, se está diciendo que no todos sus atributos
                        son necesarios para distinguir las tuplas, con lo que se contradice la
                        irreducibilidad.

                        Nótese que esta regla sólo se aplica a las relaciones propietarias


                    \item
                        \textbf{Si en una relación hay alguna Llave Foránea, sus valores deben
                        coincidir con valores de la clave primaria a la que hace referencia, o bien,
                        deben ser completamente nulos}

                \end{itemize}


            % ==============================================
            % ==========   REGLAS DE NEGOCIOS    ===========
            % ==============================================
            \subsection{Reglas de Negocio}

                Además de las dos reglas de integridad anteriores, los usuarios o los administradores
                de la base de datos pueden imponer ciertas restricciones específicas sobre los datos,
                denominadas reglas de negocio.

                Por ejemplo, si en una oficina de la empresa inmobiliaria sólo puede haber hasta veinte
                empleados, el SGBD debe dar la posibilidad al usuario de definir una regla al respecto
                y debe hacerla respetar.
                En este caso, no debería permitir dar de alta un empleado en una oficina que ya tiene
                los veinte permitidos.




        % =====================================================
        % ====   MAPEO: DE ENTIDAD RELACION A RELACIONAL   ====
        % =====================================================
        \clearpage
        \section{Mapear Entidad Relación a Relacional}

            
            % ==============================================
            % ==========      MAPEAR ENTIDADES    ==========
            % ==============================================
            \subsection{Mapear Entidades}

                \begin{itemize}
                    \item Cada entidad en un Diagrama Entidad Relación (DER) es transformado en
                        una Relación.

                    \item Cada atributo en una entidad se convierte en atributo de la relación.

                    \item El atributo identificador de la entidad se convierte en la llave primaria
                        de la relación correspondiente.
                \end{itemize}

                \begin{figure}[h]
                    \centering
                    \includegraphics[width=0.8\textwidth]{MapeoEntidad}
                \end{figure}


            % ==============================================
            % ==========      MAPEAR COMPUESTOS   ==========
            % ==============================================
            \clearpage
            \subsection{Mapear Atributos Compuestos}

                Cuando una entidad tiene atributos compuestos, solo los componentes del atributo
                compuesto son incluidos en la nueva relación.

                \begin{figure}[h]
                    \centering
                    \includegraphics[width=0.85\textwidth]{MapeoCompuesto}
                \end{figure}


            % ==============================================
            % ==========      MAPEAR MULTIVALOR   ==========
            % ==============================================
            \clearpage
            \subsection{Mapear Atributos MultiValor}

                Cuando una entidad contiene atributos multivalor, dos nuevas relaciones son creadas.

                \begin{itemize}

                    \item
                        La primera relación contiene todos los atributos de la entidad excepto el
                        atributo multivalor.
                    \item
                        La segunda relación contiene dos atributos que forman la llave primaria
                        (llave primaria compuesta) de la segunda relación.

                        El primero de estos atributos es la llave primaria de la primera relación,
                        la cual se convierte en la llave foránea en la segunda relación.
                        El segundo atributo es el atributo multivalor.
                        El nombre de la segunda relación debe ser representativo del atributo multivalor.

                \end{itemize}  

                \begin{figure}[h]
                    \centering
                    \includegraphics[width=0.85\textwidth]{MapeoMultiValor}
                \end{figure}



            % ==============================================
            % =======  MAPEAR ENTIDADES DEBILES   ==========
            % ==============================================
            \clearpage
            \subsection{Mapear de Entidades Débiles}

                Por cada par de Entidades fuertes y debiles tenemos que crear dos entidades:

                \begin{itemize}
                    
                    \item 
                        Se crea una nueva relación basada en la entidad fuerte e incluye todos los atributos
                        simples (o atributos simples de atributos compuestos) como atributos de esta relación.

                    \item Se crea otra relación basada en la entidad debíl, esta incluye la llave primaria de
                        la entidad fuerte o relación identificada como llave foránea en esta nueva relación.

                        La llave primaria de la nueva relación es la combinación de la llave primaria de la
                        entidad fuerte y el identificador parcial de la entidad débil.

                \end{itemize}

                \begin{figure}[h]
                    \centering
                    \includegraphics[width=0.95\textwidth]{MapeoEntidadDebil}
                \end{figure}



            % ==============================================
            % =======  MAPEAR RELACIONES BINARIAS  =========
            % ==============================================
            \clearpage
            \subsection{Mapear Relaciones Binarias: Uno a Uno}

                Para una relación binaria uno a uno (1:1) se debe crear 2 relaciones, una para cada entidad
                involucrada.

                \begin{itemize}
                    \item 
                        La llave primara de una de las relaciones resultantes es incluida como llave
                        foránea de la otra entidad.
                    \item
                        Si la relación binaria es opcional en una dirección y obligatoria hacia la otra dirección
                        (como se muestra en el ejemplo), la llave foránea debe de ir del lado de la relación
                        opcional, esto para evitar valores nulos.
                \end{itemize}

                \begin{figure}[h]
                    \centering
                    \includegraphics[width=0.95\textwidth]{MapeoRelacionesBinariasUnoUno}
                \end{figure}

                


            % ==============================================
            % =======  MAPEAR RELACIONES BINARIAS  =========
            % ==============================================
            \clearpage
            \subsection{Mapear Relaciones Binarias: Uno a Muchos}

                Para cada relación uno a muchos, primero se crea una relación para cada una de las dos
                entidades participantes en la relación, usando el procedimiento para mapear entidades.

                Se debe incluir la llave primaria de la entidad del lado de la cardinalidad uno como llave
                foránea en la relación que está del lado de cardinalidad muchos de la relación.

                En otras palabras, la llave primaria se pasa como llave foránea al lado de los muchos.

                \begin{figure}[h]
                    \centering
                    \includegraphics[width=0.95\textwidth]{MapeoRelacionesBinariasUnoMuchos}
                \end{figure}


            % ==============================================
            % =======  MAPEAR RELACIONES BINARIAS  =========
            % ==============================================
            \clearpage
            \subsection{Mapear Relaciones Binarias: Muchos a Muchos}

                Suponga la relación entre dos entidades A y B.
                Para hacer el mapeo se deben generar 2 relaciones una para cada una de las entidades (A y B) y
                una adicional para la relación M:N entre las entidades, que llamaremos C.

                Esta nueva relación C tendrá estas carácteristicas:

                \begin{itemize}
                    \item
                        Se debe incluir como llaves foráneas de la relación C, la llave primaria de cada
                        una de las dos entidades (A y B).
                    \item
                        Estos mismos atributos (llaves foráneas) se convierten en la llave primaria de la
                        relación C, cualquier atributo asociado a la relación M:N se debe incluir en la relación C.
                \end{itemize}

                \begin{figure}[h]
                    \centering
                    \includegraphics[width=0.95\textwidth]{MapeoRelacionesBinariasMuchosMuchos}
                \end{figure}




            % ==============================================
            % =======  MAPEAR ENTIDADES ASOCIATIVAS  =======
            % ==============================================
            \clearpage
            \subsection{Mapear Entidades Asociativas}

                Se crean 3 relaciones, una para cada una de las entidades, incluyendo la entidad asociativa
                (la cual se denomina relación asociativa al hacer el mapeo).

                Existen dos casos al hacer el mapeo al modelo relacional, el primero es si no existe un
                identificador asignado a la entidad asociativa y el segundo es si existe un identificador
                asignado a la entidad asociativa.

                \begin{itemize}
                    \item
                        Si no existe un identificador asignado, la llave primaria de la relación asociativa
                        se va a componer de dos atributos (llave compuesta), los cuales son las llaves
                        primarias de las otras dos relaciones. 

                        Los atributos simples asociados a la entidad asociativa se agregan a la relación
                        asociativa como atributos adicionales.

                        \begin{figure}[h]
                            \centering
                            \includegraphics[width=0.85\textwidth]{MapeoEntidadesAsociativas1}
                        \end{figure}

                    \clearpage

                    \item
                        Si existe un identificador asignado a la entidad asociativa, se crean tres relaciones,
                        incluyendo la relación asociativa, la llave primaria de la relación asociativa va a ser
                        el atributo identificador asociado a la misma y no será una llave compuesta como en el
                        caso anterior. Las llaves primarias de las otras dos entidades participantes son
                        incluidas como llaves foráneas en la relación asociativa resultante.

                        \begin{figure}[h]
                            \centering
                            \includegraphics[width=0.85\textwidth]{MapeoEntidadesAsociativas2}
                        \end{figure}

                \end{itemize}



            % ==============================================
            % =======  MAPEAR UNARIAS: UNO A MUCHOS  =======
            % ==============================================
            \clearpage
            \subsection{Mapear Relaciones Unarias: Uno a Muchos}

                El mapeo de las relaciones unarias sigue el mismo procedimiento que el paso 1, es decir
                se crea una relación para la entidad involucrada.

                La llave foránea es agregada como un
                atributo dentro de la misma relación que hace referencia a los valores de la llave
                primaria (esta llave foránea debe tener el mismo dominio que la llave primaria).

                \begin{figure}[h]
                    \centering
                    \includegraphics[width=0.85\textwidth]{MapeoRelacionesUnariasUnoMuchos}
                \end{figure}
                



            % ==============================================
            % ====  MAPEAR UNARIAS: MUCHOS A MUCHOS  =======
            % ==============================================
            \clearpage
            \subsection{Mapear Relaciones Unarias: Muchos a Muchos}

                Con este tipo de relación se crean dos relaciones, una para representar al tipo de entidad en
                la relación y la otra será una relación asociativa para representar la relación muchos a muchos.

                La llave primaria de la relación asociativa consistirá de dos atributos.
                Ambos atributos (los cuales no deben llevar el mismo nombre) toman los valores de la llave
                primaria de la otra relación, cualquier atributo adicional que no sea llave se incluye en la
                relación asociativa.

                \begin{figure}[h]
                    \centering
                    \includegraphics[width=0.85\textwidth]{MapeoRelacionesUnariasMuchosMuchos}
                \end{figure}



            % ==============================================
            % ====    MAPEAR RELACIONES N-ARIAS      =======
            % ==============================================
            \clearpage
            \subsection{Mapear Relaciones N-Arias}

                Una relación ternaria es una relación entre tres tipos de entidades, se sugiere que
                relaciones ternarias se conviertan en entidades asociativas para representar la participación
                de las tres entidades más apropiadamente.


                Para hacer el mapeo de una entidad asociativa que liga tres entidades se deben crear tres
                relaciones y una relación adicional, la relación asociativa.

                La llave primaria de esta relación asociativa consiste de tres atributos que son las llaves
                primarias de las entidades participantes (en ocasiones se requieren de atributos adicionales
                para formar una llave primaria única). Estos atributos también son las llaves foráneas que hacen
                referencia a las llaves primarias de las entidades que participan en la relación.

                Cualquier atributo adicional asociado a la entidad asociativa es agregado a los atributos de
                la relación asociativa.

                \begin{figure}[h]
                    \centering
                    \includegraphics[width=0.95\textwidth]{MapeoRelacionesNArias}
                \end{figure}


            % ==============================================
            % ===   MAPEAR ENTIDAD RELACION EXTENDIDO    ===
            % ==============================================
            \clearpage
            \subsection{Mapear Entidad Relación Extendido}

                \begin{figure}[h]
                    \centering
                    \includegraphics[width=0.95\textwidth]{MapeoEntidadRelacionExtendido}
                \end{figure}



        % =====================================================
        % ==========    NORMALIZACION    ======================
        % =====================================================
        \clearpage
        \section{Normalización}


            % =========================================
            % ====   RELACIONES BIEN ESTRUCTURADAS  ===
            % =========================================
            \subsection{Relación bien Estructurada}

                Es aquella que permite libremente escribir, actualizar o eliminar un registro
                sin que ocurran anomalías.

                En otras palabras una relación bien estructurada, es aquella relación que
                contiene el mínimo de redundancia y permite a los usuarios insertar, modificar
                y borrar registros en una tabla sin errores o inconsistencias.


            % =========================================
            % =======   DEFINICIÓN     ================
            % =========================================
            \subsection{Definición}

                Decimos que Normalizar es el proceso por el cual convertimos todas nuestras relaciones
                en relaciones bien estructuradas y pequeñas.
                Es decir, a efectos practicos lo que hacemos es dividir las relaciones que tenemos.


                Podemos decir alternante que Normalización es el \textbf{Proceso de eliminación de
                redundancias en una tabla para que sea más fácil de modificación}



            % =========================================
            % =======        PASO 1    ================
            % =========================================
            \clearpage
            \subsection{Paso 1: Eliminar Atributos Multivalor}

                Se debe de cumplir la Regla de Integridad de Entidades

            % =========================================
            % =======        PASO 2    ================
            % =========================================
            \clearpage
            \subsection{Paso 2: Eliminar Dependencias Funcionales Parciales}


            % =========================================
            % =======        PASO 3    ================
            % =========================================
            \clearpage
            \subsection{Paso 3: Eliminar Dependencias Transitivas}



            % =========================================
            % =======        PASO 4    ================
            % =========================================
            \clearpage
            \subsection{Paso 4: Eliminar Anomalías en Llaves Candidatas}



            % =========================================
            % =======        PASO 5    ================
            % =========================================
            \clearpage
            \subsection{Paso 5: Eliminar Dependencias Multivaluadas}


            % =========================================
            % =======        PASO 6    ================
            % =========================================
            \clearpage
            \subsection{Paso 6: Eliminar Dependencias de Junta (Join)}



                
    % ===============================================================================
    % ==============       ALMACENAMIENTO DE INFORMACIÓN       ======================
    % ===============================================================================
    \chapter{Almacenamiento de Información}


        % ==============================================
        % ===============   INDICES      ===============
        % ==============================================
        \clearpage
        \section{Indices}


            % ==============================================
            % ===========  INTRODUCCIÓN      ===============
            % ==============================================
            \subsection{Introducción}

                Muchas consultas hacen referencia solo a una pequeña proporción de los registros en un archivo.

                Por ejemplo, una consulta como \Quote{Buscar todos los instructores en el departamento de Física}
                o \Quote{Encontrar el número total de créditos obtenidos por el estudiante con ID 22201} hace
                referencia a solo una fracción de los registros del estudiante.

                Es ineficiente que el sistema lea cada tupla en la relación del instructor para verificar si
                el departamento es ineficiente para leer toda la relación estudiantil solo para encontrar la
                única tupla para la ID "32556".

                Idealmente, el sistema debería poder ubicar estos registros directamente. Para permitir estas
                formas de acceso, diseñamos estructuras adicionales que asociamos con los archivos.
                

                Un índice para un archivo en un sistema de base de datos funciona de forma muy parecida al
                índice en este libro de texto. Si queremos aprender sobre un tema en particular (especificado
                por una palabra o una frase) en este libro de texto, podemos buscar el tema en el índice al
                final del libro, encontrar las páginas donde se produce y luego leer las páginas para encontrar
                la información para la que estamos buscando. 

                Las palabras en el índice están ordenadas, lo que facilita encontrar la palabra que queremos.
                Además, el índice es mucho más pequeño que el libro, reduciendo aún más el esfuerzo necesario.

                Los índices del sistema de base de datos cumplen el mismo rol que los índices de libros en
                las bibliotecas.

                Por ejemplo, para recuperar un registro de alumno con una ID, el sistema de base de datos
                buscaría un índice para encontrar en qué bloque de disco reside el registro correspondiente,
                y luego buscará el bloque de disco, para obtener el registro de estudiante apropiado.

                Mantener una lista ordenada de ID de los estudiantes no funcionaría bien en bases de datos
                muy grandes con miles de estudiantes, ya que el índice sería muy grande; Además, aunque mantener
                el índice ordenado reduce el tiempo de búsqueda, encontrar un estudiante puede ser bastante lento.


                \clearpage

                Los índices en las BBDD pretenden aligerar las consultas, a modo de simil podríamos verlos
                como un índice de libro. Buscar un capítulo en un libro sin un índice implicaría recorrer
                el libro entero hasta que nos topasemos con el mientras que encontrarlo con un índice
                supondría recorrer este último y luego ir directamente a la página en la que se encuentra
                lo que nos interesa.

                En general podemos decir que:
                Un índice es una estructura de disco asociada con una tabla o una vista que acelera la 
                recuperación de filas de la tabla o de la vista. Un índice contiene claves generadas a
                partir de una o varias columnas de la tabla o la vista. 

                Pro:
                \begin{itemize}
                    \item Las búsquedas recorren muchas menos tuplas.
                    \item Menos carga.
                    \item Mas velocidad.
                \end{itemize}


                Contras:
                \begin{itemize}
                    \item Evidentemente los índices tienen que generarse y posteriormente actualizarse
                        con cada escritura, esto supone carga.
                    \item Los índices ocupan espacio, se que parece de perogrullo, pero cuando superan en
                        tamaño a la BBDD tienes un problema.
                \end{itemize}







% //////////////////////////////////////////////////////////////////////////////////////////////////////////
% ///////////////////////////////////         PARTE PRACTICA               /////////////////////////////////
% //////////////////////////////////////////////////////////////////////////////////////////////////////////
\part{Parte Practica}
\clearpage

    
    % ===============================================================================
    % =========================           S Q L                ======================
    % ===============================================================================
    \chapter{SQL}

        % ==============================================
        % ========      SQL COMO LENGUAJE      =========
        % ==============================================
        \clearpage
        \section{SQL como Lenguaje}
            
            SQL \textbf{no} es tan potente como una máquina universal de Turing.

            Es decir, hay algunos cálculos que son posibles utilizando un lenguaje de programación
            de propósito general, pero no son posibles con SQL.
            SQL también no admite acciones como la entrada de usuarios, la salida a las pantallas 
            ó la comunicación a través de la red.
            Dichas computaciones y acciones deben escribirse en un lenguaje principal (C, C++ ó Java)
            con consultas SQL incorporadas que acceden a los datos de la base de datos

            Los programas de aplicación son programas que se utilizan para interactuar con la base de datos de esta manera.



    % ===============================================================================
    % ===================        QUERIES EN SQL                ======================
    % ===============================================================================
    \chapter{Queries en SQL}
    \clearpage

        % ========================================
        % =========   SELECT FROM    =============
        % ========================================
        \section{SELECT FROM}

            Empecemos por la sentencia más básica para un Query en SQL:

            \begin{lstlisting}[language=SQL, gobble=16]
                SELECT FielName FROM TableName;
            \end{lstlisting}

            Ahora empecemos por aquí:
            \subsection{Distint vs All}
            \begin{itemize}
                \item Distint (Default):
                    Esta solo muestra solo los atributos diferentes

                    \begin{lstlisting}[language=SQL, gobble=24]
                        SELECT DISTINT FielName FROM TableName;
                    \end{lstlisting}
                \item All:
                    Esta muestra todos los atributos, incluso duplicados
                    \begin{lstlisting}[language=SQL, gobble=24]
                        SELECT ALL FielName FROM TableName;
                    \end{lstlisting}
            \end{itemize}



        % ========================================
        % =========   COUNT , AVE, SUM  ==========
        % ========================================
        \section{Funciones de Agregación}

            Las funciones de agregación son funciones que toman una colección
            como entrada y regresan un simple valor. SQL ofrece estas 5 funciones
            por default:

            \begin{itemize}
                \item \Quote{AVG()}: Promedio
                \item \Quote{MIN()}: Minimo
                \item \Quote{MAX()}: Máximo
                \item \Quote{COUNT()}: Cuenta la cantidad 
                \item \Quote{SUM()}: Suma todo el resultado
            \end{itemize}





        % ========================================
        % ============     WHERE   ===============
        % ========================================
        \clearpage
        \section{WHERE}

            Esta nos permite seleccionar solo las filas en el resultado del quey
            que satisface nuestros criterios.

            \begin{lstlisting}[language=SQL, gobble=16]
                SELECT FielName FROM TableName
                    WHERE
                        (TableName.Thing1 < Something) AND
                        (TableName.Thing2 = "Some Strings");
            \end{lstlisting}

            Tenemos a nuestra dispoción los siguientes comparadores:
            \begin{itemize}
                \item Lógicos: AND, OR, NOT
                \item Aritmeticos: <, >, <=, >=, =, <>
            \end{itemize}



        % ========================================
        % ============   STRINGS   ===============
        % ========================================
        \section{Comparación de Strings}
            
            Podemos comparar strings gracias a lo que conocemos como comodínes usando
            el comando \Quote{LIKE}
            \begin{itemize}
                \item \Quote{\%}: Que sirve como comodín para cualquier cantidad de caracteres 
                \item \Quote{\_}: Que sirve como comodín para un caracter 
            \end{itemize}


        % ========================================
        % ========   ORDEN DE TUPLAS   ===========
        % ========================================
        \section{Orden de las Tuplas}

            Podemos controlar el orden en que se nos retornan las tuplas del query que estamos
            haciendo muy facilmente con la instructiva ORDER BY.

            Por default se usa la opción \Quote{ASC} que las ordena de manera ascendente
            mientras que \Quote{DESC} la ordena de manera descendente.

            Por default esta se coloca la última línea de manera implicita:
            \begin{lstlisting}[language=SQL, gobble=16]
                SELECT DISTINT FielName1, FielName2, FielName3 FROM TableName
                    ORDER BY FielName1 ASC, FielName2 ASC, FielName3 ASC;
            \end{lstlisting}

            También podemos simplemente poner el número en que las pusimos para 
            indicar el orden:

            \begin{lstlisting}[language=SQL, gobble=16]
                SELECT DISTINT FielName1, FielName2, FielName3 FROM TableName
                    ORDER BY 1,2, 3;
            \end{lstlisting}


    % ===============================================================================
    % ===================        VIEWS EN SQL                  ======================
    % ===============================================================================
    \chapter{Views en SQL}


        En teoría de bases de datos, una vista es una consulta que se presenta como una tabla (virtual)
        a partir de un conjunto de tablas en una base de datos relacional.

        Las vistas tienen la misma estructura que una tabla: filas y columnas.
        La única diferencia es que sólo \textbf{se almacena de ellas la definición, no los datos}.
        Los cambios aplicados a los datos en una tabla se reflejan en los datos mostrados en invocaciones
        posteriores de la vista.
        En algunas bases de datos NoSQL, las vistas son la única forma de consultar datos.

        Los datos que se recuperan mediante una consulta a una vista se presentarán igual que los de una tabla.

        Una vista se crea a través de una expresión de consulta (una sentencia SELECT) que la calcula y que puede
        realizarse sobre una o más tablas.


        % ========================================
        % =========     VENTAJAS     =============
        % ========================================
        \section{Ventajas}

            Las vistas pueden ofrecer ventajas sobre las tablas:

            \begin{itemize}

                \item
                    Las vistas pueden representar un subconjunto de los datos contenidos en una tabla.
                    En consecuencia, una vista puede limitar el grado de exposición a las tablas al mundo exterior:
                    un usuario dado puede tener permiso para consultar la vista, mientras que se deniega el acceso
                    al resto de la tabla base.

                \item
                    Las vistas pueden ocultar la complejidad de los datos.
                    Por ejemplo, una vista podría aparecer como Sales2000 o Sales2001, particionando
                    de forma transparente la tabla subyacente real.

                \item 
                    Las vistas cuestan muy poco espacio para almacenar; la base de datos contiene sólo
                    la definición de una vista, no una copia de todos los datos que presenta.

            \end{itemize}



        % ========================================
        % =========    SINTAXIS      =============
        % ========================================
        \section{Sintaxis}

            Empecemos por la sentencia general de las vistas:

            \begin{lstlisting}[language=SQL, gobble=16]
                CREATE VIEW ViewName AS
                    SELECT FielName
                        FROM TableName
                        WHERE Conditions
                        ORDER BY FielName; 
            \end{lstlisting}

    % ===============================================================================
    % ===================        PROCEDURES EN SQL               ====================
    % ===============================================================================
    \clearpage
    \chapter{Procedures en SQL}

        Literalmente son como funciones, bueno, más específicamente son procedimientos,
        es decir un conjunto de instrucciones en SQL que se ejecutan una tras otra.


        % ========================================
        % =========    SINTAXIS      =============
        % ========================================
        \section{Sintaxis}

            Empecemos por la sentencia general de las vistas:

            \begin{lstlisting}[language=SQL, gobble=16]
                DELIMITER &
                CREATE PROCEDURE ProcedureName()
                BEGIN
                    SQLInstruccion1;

                    SQLInstruccion2;

                    ...

                    SQLInstruccionN;
                END &

                DELIMITER ;
                CALL ProcedureName;
            \end{lstlisting}

\end{document}




























